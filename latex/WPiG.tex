% !TEX encoding = UTF-8 Unicode

\documentclass[12pt]{amsbook}

\usepackage{agda}

\usepackage{geometry}
\geometry{b5paper}

\usepackage{bbm}
\usepackage[greek,english]{babel}

\usepackage{hyperref}

% \usepackage{ucs}
% \usepackage[utf8x]{inputenc}
% \usepackage{autofe}

% \usepackage{fancyvrb}
% \DefineVerbatimEnvironment
%   {code}{Verbatim}
%   {}

\setmainfont{XITS}
\setmathfont{XITS Math}

\usepackage[xindy]{glossaries}
\makeglossaries
\newglossaryentry{alphabet}
{
  name=alphabet,
  description={nothing more than a finite set}
}

\newglossaryentry{binary operation}
{
  name={binary operation},
  description={TODO: write description}
}

\newglossaryentry{concatenation}
{
  name=concatenation,
  description={TODO: write description}
}

\newglossaryentry{identity element}
{
  name={identity element},
  description={a special type of element of a set, with respect to a
  \gls{binary operation} on that set, which leaves other elements unchanged
  when combined}
}


\newglossaryentry{length}
{
  name=length,
  description={TODO: write description}
}

\newglossaryentry{letter}
{
  name=letter,
  description={an element of an \gls{alphabet}}
}

\newglossaryentry{monoid}
{
  name=monoid,
  description={\glossentrydesc{semigroup} and an identiy}
}

\newglossaryentry{semigroup}
{
  name=semigroup,
  description={a set with an associative multiplication}
}


\newglossaryentry{nullstring}
{
  name=nullstring,
  symbol={$\epsilon$},
  description={a unique \gls{string} over an \gls{alphabet} $A$ where $n = 0$
  and the domain is the nullset, generally denoted \glssymbol{nullstring}}
}


\newglossaryentry{string}
{
  name=string,
  description={a finite sequence of \glspl{letter}, i.e. an integer $n \geq 0$
  and a mapping $\{1,...,n\} \to A$}
}

\renewcommand*{\glstextformat}[1]{\textsl{#1}}


\usepackage{todonotes}

\title{Word Processing in Groups}
\author{Eric Bailey}
% \date{}

\usepackage{outlines}

\begin{document}

\maketitle
\tableofcontents

\newpage
\part{An Introduction to Automatic Groups}

\chapter{Finite State Automata, Regular Languages and Predicate Calculus}

\section{Languages and Regular Languages}

\begin{outline}
  \1 An \gls{alphabet} $A$ is \glossentrydesc{alphabet}.
  \2 If $A$ is the \gls{alphabet} over lowercase letters, ``automaton'' is a
  \gls{string} over $A$ with $n = 9$.
  \2 If $\omega$ is a \gls{string} $\{1,...,n\} \to A$, we call $n$ the
  \gls{length} of $\omega$ and we denote it by $|\omega|$.
  \1 An element of a $A$ is called a \gls{letter}.
  \1 A \gls{string} over the \gls{alphabet} $A$ is \glossentrydesc{string}.
  \1 If $n = 0$, the domain is the nullset and there is a unique string, the
    \gls{nullstring}, generally denoted \glsentrysymbol{nullstring}, or
    sometimes $\epsilon_A$ to distinguish the \gls{nullstring} over $A$, since
    $\epsilon$ might be a \gls{letter} in $A$, e.g. Definition 1.1.3.
    %% TODO: link def 1.1.3
  \1 The set of all \glspl{string} over the \gls{alphabet} $A$ is denoted
  $A^*$.
  \2 With the operation of \gls{concatenation}, the set $A^*$ of \glspl{string}
  over $A$ forms a \gls{monoid}, with \gls{identity element}
  \glsentrysymbol{nullstring}.
  \2 $A^*$ is the free monoid or \gls{semigroup} on the set of generators $A$.
  \1 All \glspl{semigroup} considered in this book will be \glspl{monoid}, so
  the words are used interchangeably.
  \1 Given two \glspl{string} $\omega : \{1,...,n\} \to A$ and $\tau :
  \{1,...,m\} \to A\}$, the \gls{concatenation} $\omega\tau$ of $\omega$ and
  $\tau$ is defined to be the \gls{string} $\{1,...,m+n\} \to A$ given by
  $(\omega\tau)(i) = \omega(i)$ if $1 \leq i \leq n$ and $(\omega\tau)(i) =
  \tau(i-n)$ if $n + 1 \leq i \leq n + m$.
\end{outline}

\newpage
\printglossaries

\end{document}